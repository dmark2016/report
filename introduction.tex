\section{Introduction}
\label{sec:introduction}

In modern computer architecture, the memory hierarchy plays an important part.
Without intelligent usage of caches and scheduling of requests to main memory, the processor will spend an undesirable amount of time idle while waiting for data.
Prefetching aims to reduce cache misses and increase data availability by attempting to predict which data will be needed in the future and fetch it early. \todo{rephrase}
Designing a prefetcher that performs well for all programs is difficult.
Depending on the type of program, different memory access patterns arise, making it necessary for the prefetcher to be adaptive.

In this paper, we present a prefetcher design that measurably increases performance for a wide range of program types.

Our final prefetcher is an implementation of global history buffer with delta correlation.
The memory accesses made by the instructions are stored in a linked list.
When an instruction is encountered again, the linked list is traversed and the deltas between the earlier accesses are examined to see wether a pair of deltas equal to the most recent ones exist.
If it does, a block is prefetched using the same delta.

This paper provides background (section~\ref{sec:background}) on various prefetcher strategies as well as an overview of our process and methodology (section~\ref{sec:methodology}).
Subsequently, the finished prefetcher design is presented (section~\ref{sec:prefetcher}), followed by results and perfomance details (section~\ref{sec:results}).
Finally, we discuss (section~\ref{sec:discussion}) our findings, provide an overview of some related works (section~\ref{sec:related-work}) and a conclusion (section~\ref{sec:conclusion}).
\todo{better way of writing this?}

%The introduction section introduces the larger research area
%the paper is a part of and illustrates the concrete problem(s) at
%hand the paper tries to solve. It explains the proposed solution
%from a 20.000 feet abstraction level. Furthermore, it states
%the contributions of the paper and briefly highlights its main
%results. It finishes with an outline of the paper, giving a short
%explanation of the contribution/meaning of each section.
